
\documentclass{article}
\usepackage[a4paper,left=3.5cm,right=2.5cm,top=2.5cm,bottom=2.5cm]{geometry}
%%\usepackage[MeX]{polski}
\usepackage[cp1250]{inputenc}
%%\usepackage{polski}
%%\usepackage[utf8]{inputenc}
\usepackage[pdftex]{hyperref}
\usepackage{makeidx}
\usepackage[tableposition=top]{caption}
\usepackage{algorithmic}
\usepackage{graphicx}
\usepackage{enumerate}
\usepackage{multirow}
\usepackage{amsmath} %pakiet matematyczny
\usepackage{amssymb} %pakiet dodatkowych symboli
\begin{document}
	\begin{displaymath}
		\sum_{i=0}^{\infty}{2^i}
	\end{displaymath}

	\begin{equation}
		\label{eq:iloczyn}
		\prod^{n=i^2}_{i=2}=\frac{\lim^{n\rightarrow4} (1+\frac{1}{n})^n }{\sum k (\frac{1}{n})}
	\end{equation}
	
	\begin{equation}
			\label{eq:lim}
			\lim_{n\rightarrow\infty}\sum_{k-1}^{n}{\frac{1}{k^2}=\frac{\pi^2}{6}}
	\end{equation}
	
	\begin{equation}
		\label{eq:16}
		\left[x\right]_A=\{y\in U :a(x) = a(y),\forall_a \in A\}, where the central object x\in U
	\end{equation}
	
	\begin{equation}
		\label{eq:20}
		cos(2\theta) = cos^2 - sin^2\theta
	\end{equation}
	
	\begin{equation}
		\label{eq:23}
		P$\Big(A=2|\frac{A^2}{B}>4\Big)$
	\end{equation}

	\begin{equation}
	\left[
	\begin{array}{cccc}
	a_{11} & a_{12} & \ldots & a_{1K} \\
	a_{21} & a_{22} & \ldots & a_{2K} \\
	\vdots & \vdots & \ddots & \vdots \\
	a_{K1} & a_{K2} & \ldots & a_{KK} \\
	\end{array}
	\right]
	*
	\left[
	\begin{array}{c}
	x_{1} \\
	x_{2} \\ 
	\vdots \\ 
	x_{K} \\
	\end{array}
	\right]
	=
	\left[
	\begin{array}{c}
	b_{1} \\
	b_{2} \\ 
	\vdots \\ 
	b_{K} \\
	\end{array}
	\right]	
	\end{equation}
	
	\begin{verbatim}
	for(int i=0; i<40;i++)
		printf("Hello World");
	\end{verbatim}
	
	\begin {algorithmic}
	\FOR{i=0,1,$\ldots$, 40}	
		\item{Wy�wietl napis "Helloworld"}
	\ENDFOR
	\end {algorithmic}
	
	
	Tu umieszczamy kod TeXa, ktory bedzie kompilowany, $a^2$ a suma $\sum_{i=0}^{\infty}{2^i}$
\end{document}